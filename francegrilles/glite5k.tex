\documentclass[a4paper,11pt]{article}
\usepackage{a4wide}
\usepackage[french]{babel}
\usepackage[utf8]{inputenc}  %or: \usepackage[latin1]{inputenc}
\usepackage{color}
\usepackage{hyperref}

\newcommand{\todo}[1]{{\color{red}\textsl #1}}


\title{gLite sur Grid'5000:\\ vers une plate-forme d'expérimentation\\ à taille réelle pour les grilles de production}
\date{}
\author{Sébastien Badia et Lucas Nussbaum\\
\texttt{\normalsize \{sebastien.badia,lucas.nussbaum\}@loria.fr}\\[1em]
\normalsize ALGORILLE team, LORIA\\
INRIA Nancy - Grand Est \& Nancy-Université}

\begin{document}
\maketitle

\section{Overview}
\todo{Résumé en anglais}

The Grid has become a huge and important project, playing a key role in the
everyday work of many researchers. A large amount of software is being developed
both to manage the grid infrastructure itself (gLite middleware\cite{glite}),
to facilitate the task of grid users (e.g workflow managers\cite{moteur},
pilot job managers, etc.), and to run the computations themselves.
That software must be designed to handle network and services outages in a
highly distributed environment, while still providing the expected performance.
It is inconvenient to test software using the production infrastructure, since
(1) it might not exhibit the behaviour that is required to test extreme
conditions (services are unlikely to crash as often as required when testing
fault tolerance); (2) it might not be possible to replace key parts of the
infrastructure without degrading the user experience.

In this paper, we present our ongoing work on deploying the gLite middleware on
the Grid'5000 testbed. Grid'5000 is a scientific instrument designed to support
research on parallel, large-scale and distributed computing. It is composed of
FIXME nodes (FIXME cores) with unique hardware and network reconfiguration
features: users can reserve nodes, and then automatically re-install them with
the software environment required for their experiments.

% FIXME



\section{Enjeux scientifiques}
\todo{Enjeux scientifiques, besoin de la grille : Environ une demi-page}

La devenue est devenue une immense et importante plate-forme, qui joue un rôle
clé dans la travail quotidien de nombreux chercheurs. Un grand nombre de
logiciels ont été développés pour gérer l'infrastructure elle-même (middleware
gLite\cite{glite}), pour faciliter le travail des utilisateurs (moteurs de
workflows comme MOTEUR~\cite{moteur}, gestionnaires de jobs pilotes, etc.), et
pour exécuter les traitements eux-mêmes.
Ces logiciels doivent être conçus pour prendre en compte les pannes à
différents niveaux (réseaux, services), dans une infrastructure largement
distribuée, tout en fournissant les performances requises.

Il est en général difficile de tester ces logiciels en utilisant
l'infrastructure de production. D'une part, il est peu probable que
l'infrastructure de production fournisse le comportement souhaité pendant les
tests : lors du test de la résistance aux pannes d'un logiciel, il serait peu
confortable d'attendre une panne sur l'infrastructure de production pour
vérifier son bon fonctionnement. D'autre part, il n'est pas forcément possible
de remplacer un composant central de l'infrastructure pour en tester une
modification sans gêner les utilisateurs.

Dans ce travail, nous proposons d'utiliser la plate-forme
Grid'5000~\cite{grid5000,grid5000web}, dédiée à la recherche sur les systèmes
et le calcul parallèle, pour tester l'infrastructure logicielle des grilles de
production.  Nous visons les cas d'utilisation suivants :
\begin{description}
\item[foo]
\item[bar]
\item[baz]
\end{description}

\section{Développements}
\todo{Développements, déploiement sur la grille : Environ une demi-page}

La plate-forme Grid'5000 est composée de 1700 machines (7000 coeurs) répartis
dans 10 sites en France. Elle dispose de fonctionnalités de reconfiguration du
matériel et du réseau : les utilisateurs peuvent, après avoir réservé des
ressources, réinstaller les machines réservées avec le système requis pour leur
expérience.

Cette fonctionnalité a été utilisée pour développer un ensemble de scripts
permettant d'automatiser le déployement d'une infrastructure gLite sur
Grid'5000, composée de:
\begin{itemize}
\item une VO et son VOMS (\textsl{Virtual Organization Membership Service}), annuaire des utilisateurs ;
\item plusieurs sites, composés de:
\begin{itemize}
	\item un BDII (\textsl{Berkeley Database Information Index}), annuaire des ressources disponibles sur chaque site ;
	\item un CE (\textsl{Computing Element}), interface de soumission des tâches ;
	\item un WMS (\textsl{Workload Management System}); dans le cadre de notre travail, Torque a été utilisé ;
	\item une UI (\textsl{User Interface}), interface d'accès pour les utilisateurs ;
	\item un ou plusieurs clusters composés de noeuds de calcul.
\end{itemize}
\end{itemize}

\section{Outils}
\todo{Outils, difficultés rencontrées : Environ une demi-page}

\section{Résultats scientifiques}
\todo{Résultats scientifiques : Environ une demi-page à trois quarts de page}

\begin{verbatim}
2 déploiements:
- l'un en matchant les sites \& clusters Grid'5000
- l'autre en "unifiant" les ressources, et en déployant un grand
  nombre de VO (50?) / clusters
+ infos de timing

Q: terminologie gLite: VO vs site, etc?
Q: liens vers les meilleures documentations utilisées?
\end{verbatim}

\section{Perspectives}
\todo{Perspectives : Environ une demi-page}

\section{Références}
\todo{Références : Environ une demi-page}

\begin{verbatim}

http://graal.ens-lyon.fr/~desprez/FILES/ProdRech.html
\end{verbatim}


\end{document}

