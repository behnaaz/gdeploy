\documentclass[a4paper, 11pt]{article}

\usepackage{a4wide}
\usepackage{url}
\usepackage{xspace}
\usepackage[utf8]{inputenc}
\usepackage{paralist}

\title{Grid-in-a-grid:\\ Deployment of a gLite Grid inside Grid'5000}
\author{Sébastien Badia and Lucas Nussbaum\\\texttt{\small \{sebastien.badia,lucas.nussbaum\}@loria.fr}\\[0.5em]LORIA / INRIA Nancy -- Grand Est}
\date{}

\begin{document}
\maketitle
\section{Context}

The gLite grid middleware is the core software component of the EGI (formerly
EGEE) production grid, composed of more than 330 sites and 200000 CPUs.  gLite
is organized as a set of services to manage security and authentication
(Virtual Organisation Membership Service -- VOMS), interact with the local
batch schedulers (Computing Element -- CE), manage data at the grid level
(Storage Element -- SE), and expose information on the available resources
(Information Service -- IS). It also provides a user interface to the users,
enabling them to find resources, submit or cancel jobs, show the status or the
output of jobs, manage their files, etc.

The performance and ease of use of the grid platform is highly dependent on
this software stack, and it is therefore crucial to be able to evaluate
possible improvements in a test environment. Unfortunately, building such a
test environment with the required features is very hard, as it needs to be
flexible enough to enable developers to replace some parts of the
infrastructure, and large enough to reproduce problems that only arise when a
very large number of resources are involved.  Currently, engineers and
researchers directly use the production platform to test their improvements,
which has several limitations (risk of breaking the infrastructure, low
reproducibility of experiments, waste of production resources).

Grid'5000 is a perfect basis for a development and test environment for
production grid software such as gLite and other related services. It has most
of the required features, being composed of a large number of nodes grouped in
clusters and sites which matches (at a smaller scale) the architecture of the
EGI production grid, and providing users with the ability to deploy their own
software stack. However, a lot of hard work is needed to overcome the technical
locks that one encounters when trying to deploy such complex software.

The goal of this challenge submission is to demonstrate the automated
deployment of a gLite grid inside Grid'5000. This is a required first step
towards more intense collaboration between the \textsl{production Grid} and the
\textsl{research on grids} communities.

\section{Description of the experiment and expected demo}

This experiment includes the deployment of the gLite middleware on several
Grid'5000 sites, and the setup of the various services composing a gLite
infrastructure.

We chose to deploy the gLite middleware with a setup matching closely the way
it is installed on EGI. Therefore, instead of using the standard Debian-based
Grid'5000 images, we created a \textsl{Scientific Linux 5.5} image for
Kadeploy.

Our script, written in Ruby, is executed from a central place (a user frontend,
typically).  After being started, it will first reserve and deploy the
resources using the Grid'5000 API. Then, it will configure the various pieces
of the gLite infrastructure that we are going to deploy. To do this
configuration, two main tools are used:

\begin{compactitem}

\item the \texttt{net/ssh} and \texttt{net/scp} Ruby libraries are used to
	connect to nodes and copy files on them

\item for some large files, instead of using \texttt{net/scp}, we instead make
	use of the \texttt{\~/public} directories that are exported via HTTP on
	all Grid'5000 sites.

\end{compactitem}

The infrastructure that will be deployed is composed of:

\begin{compactitem}

\item one BDII service (\textsl{Berkeley Database Information Index}). This
	service is a directory service based on LDAP containing the description
	of the various resources and services available on the grid.

\item one Computing Element (CE) and one batch scheduler per Grid'5000 cluster.
	The batch scheduler (Torque in our case) is responsible for scheduling
	the jobs on the worker nodes. The Computing Element (CE) is an
	abstraction layer enabling users to submit jobs in a way that is
	independent of the batch scheduler being used.

\item worker nodes, for each other available Grid'5000 node, where the jobs are
	executed.

\end{compactitem}

Our scripts are available in a Git repository at \url{https://github.com/sbadia/gdeploy}.

\section{Future work}

Automating the deployment of gLite is not an easy task, and several hard
technical challenges already had to be overcome.  For example, during the development,
the gLite version changed on several occasions, breaking dependencies between
packages or simply breaking our scripts. Unfortunately, there is more to come.

\subsection*{Generic Scientific Linux image}

It is harder to generate an image for Scientific Linux that works on every
Grid'5000 cluster, than it is for Debian images. For example, our image is
currently limited to clusters using the AHCI interface to control the SATA bus.
Working on a more generic Scientific Linux image is a prerequisite to perform
experiments at a larger scale.

\subsection*{Other gLite services}

Our script currently deploys only a subset of gLite services. The next step is
to install the UI service (which acts as a proxy between the user and the
various CE). Its installation was not manually, but still needs to be
automatized.

Another step is to enable a deployment of a more complex grid, with several
Virtual Organizations (VO). For example, we could deploy a gLite grid with one
VO per Grid'5000 site. This requires dealing with certificates, which are
central to the gLite security model, and also very hard to automatize.

Finally, one aspect that has been neglected so far is storage. On EGI, most
jobs are actually much more data-intensive than CPU-intensive, and the handling
of data plays a crucial role in the performance of grid applications. gLite
includes several services that deal with storage (LFC, Storage Element (SE)).
They will also have to be deployed.

\subsection*{Performing real experiments on gLite}

The deployment of gLite on Grid'5000 is of limited interest if it is not used
by real experiments to improve the middleware itself, work on software layers
on top of the middleware, or explore research questions on production grids.
This will require designing and orchestrating complex experiments, involving
complex software and a large number of resources. Further developments of
experimentation software are likely to be required in order to perform such
experiments without sacrificing experimenter time, or experimental quality.

\end{document}
